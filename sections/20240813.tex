\section*{2024.08.13 不限时测验}
如无特殊说明,设$\bbk$是数域,$m,n$是正整数。
\begin{enumerate}
    \item \begin{enumerate}
        \item 对于正实数$a_1,\dots,a_n$, 证明\begin{equation*}
            n/\sum_{k=1}^{n}\frac{1}{a_k} \leqslant \Bigl(\prod_{k=1}^{n}a_k\Bigr)^{1/n} \leqslant \frac{1}{n}\sum_{k=1}^{n}a_k.
        \end{equation*}
        \item 设$0<a,b,p<\infty$. 证明:存在只依赖于$p$的正常数$c=c(p), C=C(p)$, 使得\begin{equation*}
            c\cdot(a+b)^p\leqslant a^p+b^p\leqslant C\cdot(a+b)^p.
        \end{equation*}
        \item 设$a,b>0,1<p,q<\infty,\frac{1}{p}+\frac{1}{q}=1$. 试证明如下形式的Young不等式:\begin{equation*}
            ab\leqslant \frac{a^p}{p}+\frac{b^q}{q}.
        \end{equation*}
        提示:可以利用指数或对数函数的凹凸性。
    \end{enumerate}
    \item 计算\begin{equation*}
        \frac{\differential}{\differential x}\int_{0}^{x}\mathrm{e}^{t}\sin \left((x-t)^2\right)\differential t.
    \end{equation*}
    \item 设实数列$\left(a_n\right)_{n=1}^\infty$满足递推关系$a_{n+1}=f(a_n),n=1,2,3,\dots.$ 试证明:\begin{enumerate}
        \item 如果$f$单调递增,那么$\left(a_n\right)_{n=1}^\infty$单调。
        \item 如果$f$单调递减,那么$\left(a_n\right)_{n=1}^\infty$的奇偶子列各自都是单调的,但具有相反的单调性。
    \end{enumerate}
    \item 给定正数$a_0,b_0$, 然后按照递推式\begin{equation*}
        a_n=(a_{n-1}+b_{n-1})/2,b_n=(a_{n-1}b_{n-1})^{1/2},n=1,2,3,\dots
    \end{equation*}定义数列$\left(a_n\right)_{n=0}^\infty,\left(b_n\right)_{n=0}^\infty$. 证明$\left(a_n\right)_{n=1}^\infty,\left(b_n\right)_{n=1}^\infty$收敛到同一个极限值。
    \item 设有正数列$\{a_n\}_{n=1}^{\infty}$, 试证明\begin{equation*}
        \liminf_{n\to\infty}\frac{a_{n+1}}{a_n}\leqslant\liminf_{n\to\infty}\sqrt[n]{a_n}.
    \end{equation*}
    \item 设非负值函数$f\in C[a,b]$. 试证明\begin{equation*}
        \lim_{n\to\infty}\Bigl(\int_a^b f(x)^n \differential x\Bigr)^{1/n}=\max_{a\leqslant x\leqslant b}f(x).
    \end{equation*}
    \item 设有映射$f\colon\realset^n\to\realset^m$. 试证明:$f$连续当且仅当对于任意$\realset^m$中的开集$O$, 都有$f^{-1}(O)$是$\realset^n$中的开集。
    \item 设$A\in M_{m\times n}(\bbk)$. 证明:$\rank A\leqslant r$当且仅当存在$u\in M_{m\times r}(\bbk),v\in M_{r\times n}(\bbk)$, 使得$A=uv$.
    \item 设$A\in M_{n\times n}(\bbk)$. 证明:如果$\rank A^m=\rank A^{m+1}$, 那么$\left(\rank A^k\right)_{k=m}^\infty$为常值序列。
    \item 设$A\in M_{4\times 4}(\bbk)$. \begin{enumerate}
        \item 如果$\tr A^k=k,k=1,2,3,4$, 求$\det A$.
        \item 试证明:$A^4=0$等价于$\tr A^k=0,k=1,2,3,4$.
    \end{enumerate}
    \item 设$A\in M_{n\times n}(\bbk)$, $\rank A=1$, $\tr A=t$.
    \begin{enumerate}
        \item 试表达出$A$的特征多项式。
        \item 说明$A$在$\bbk$上一定有Jordan标准形(即一定相似于某个Jordan形矩阵),并讨论$A$在$\bbk$上的Jordan标准形。
        \item 试给出$A$的最小多项式。
    \end{enumerate}
    \item 设$A,B\in M_{n\times n}(\complexset)$没有公共特征值。试证明:\begin{enumerate}
        \item 如果某个$X\in M_{n\times n}(\complexset)$满足$AX=XB$, 那么对于任何$f(x)\in\complexset[x]$, 都有\begin{equation*}
            f(A)X=Xf(B).
        \end{equation*}
        \item 对于任何$C\in M_{n\times n}(\complexset)$, 关于$X$的矩阵方程\begin{equation*}
            AX-XB=C
        \end{equation*}都在$M_{n\times n}(\complexset)$中存在唯一解。
    \end{enumerate}
    \item 设$A\in M_{n\times n}(\bbk)$. 记$C(A):=\{B\in M_{n\times n}(\bbk)\colon AB=BA\},\bbk[A]:=\{f(A)\in M_{n\times n}(\bbk)\colon \allowbreak f(x)\in\bbk[x]\}$. $A$的特征多项式设为$f(\lambda)=\lambda^n+a_{n-1}\lambda_{n-1}+\dots+a_1\lambda+a_0\in\bbk[\lambda]$, 最小多项式设为$m(\lambda)=\lambda^s+b_{s-1}\lambda_{s-1}+\dots+b_1\lambda+b_0\in\bbk[\lambda]$, 其中$s$是一个正整数。
    \begin{enumerate}
        \item 验证$C(A)$和$\bbk[A]$都是$\bbk$上的线性空间。
        \item 求$\dim \bbk[A]$.
        \item 证明:如果$m(\lambda)$在$\bbk[\lambda]$上不可约,那么$\bbk[A]$中的任一非零矩阵都可逆。
    \end{enumerate}
\end{enumerate}
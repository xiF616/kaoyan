\section*{2024.11.28}
\begin{enumerate}
    \item 固定$x>0$, 给出
    \begin{equation*}
        a_n:=\prod_{k=1}^n (x+k)
    \end{equation*}在$n\to \infty$时的一个渐进估计,并判敛
    \begin{equation*}
        \sum_{n\geqslant 1}\frac{n!}{a_n}.
    \end{equation*}
    \item 给出\begin{equation*}
        a_n:=\frac{(2n-1)!!}{(2n)!!}\cdot\frac{1}{2n+1}
    \end{equation*}在$n\to \infty$时的一个渐进估计,并判敛
    \begin{equation*}
        \sum_{n\geqslant 1}a_n.
    \end{equation*}
    \item 设连续函数$f\colon \realset\to\realset$以$p$为周期,证明
    \begin{equation*}
        \lim_{x\to +\infty}\frac{1}{x}\int_{0}^{x}f=\frac{1}{p}\int_{0}^{p}f.
    \end{equation*}
    \item 计算
    \begin{equation*}
        \frac{\differential}{\differential x}\begin{vmatrix}
            x-1 & 1 & 2 \\ -3 & x & 3 \\ -2 & -3 & x+1
        \end{vmatrix}.
    \end{equation*}
    \item 将正切函数Maclaurin展开到5次项,带Peano余项即可。
    \item 讨论如下$\realset^2\to\realset^2$的函数的连续性:
    \begin{equation*}
        (x,y)\mapsto \begin{cases}
            y^2\ln (x^2+y^2), & x^2+y^2\neq 0, \\
            0, & x^2+y^2= 0.
        \end{cases}\quad\text{和}\quad(x,y)\mapsto \exp\bigl(-\frac{x}{y}\bigr).
    \end{equation*}
    \item 设$r(x,y,z)=\sqrt{x^2+y^2+z^2}$, 求\begin{equation*}
        \grad r\qquad\quad\text{和}\qquad\quad\grad \frac{1}{r}.
    \end{equation*}
    \item 证明
    \begin{equation*}
        \realset\to\realset,\;y\mapsto \int_{0}^{1}\sgn (x-y)\differential x
    \end{equation*}
    在$\realset$上连续。
    \item 计算
    \begin{equation*}
        \int_{0}^{\infty}\mathrm{e}^{-x}\frac{1-\cos(xy)}{x^2}\differential x.
    \end{equation*}
    \item 求
    \begin{equation*}
        \inf_{a,b\in \realset}\int_{1}^{3}(a+bx-x^2)^2\differential x.
    \end{equation*}
    \item 求质量分布均匀的摆线
    \begin{equation*}
        x=t-\sin t,\,y=1-\cos t\quad(0\leqslant t\leqslant \pi)
    \end{equation*}的质心。
    \item 设$a>0,k>0$, 对于$L\colon r=a\mathrm{e}^{k\theta},r\leqslant a$, 计算
    \begin{equation*}
        \int_L x\differential s.
    \end{equation*}
    \item 对于$V\colon x\geqslant 0,y\geqslant 0,0\leqslant z\leqslant 1,x^2+y^2\leqslant 1$, 计算
    \begin{equation*}
        \iiint_V xy+yz+zx \differential x\differential y\differential z
    \end{equation*}
    % \item 设$S$是$\realset^3$中的封闭曲面,计算
    % \begin{equation*}
        
    % \end{equation*}
    \item 设$f\colon \realset\to\realset$是连续函数,$D:=\{(\theta,\varphi)\colon 0\leqslant \theta\leqslant 2\pi,0\leqslant \varphi\leqslant \pi\}$, $m,n,p$不同时为0, 证明
    \begin{equation*}
        \iint_D f(m\sin\varphi\cos \theta+n\sin\varphi\sin\theta+p\cos\varphi)\sin\varphi\differential \theta\differential \varphi=2\pi\int_{-1}^{1}f\bigl(u\sqrt{m^2+n^2+p^2}\bigr)\differential u.
    \end{equation*}
    \item 记$\boldsymbol{r}=(x,y,z)$, $r=\sqrt{x^2+y^2+z^2}$, $V$是$\realset^3$中的区域,$\boldsymbol{n}$是$\partial V$的外法向量,证明
    \begin{equation*}
        \iiint_V \frac{1}{r}\differential x\differential y\differential z=\frac{1}{2}\oiint_{\partial V}\cos (\widehat{\boldsymbol{r},\boldsymbol{n}})\differential S.
    \end{equation*}
    \item 设$f,f_n(n\in\naturalset)$都是可测集$E$上的几乎处处有限的函数,并且$mE(f_n\neq f)<2^{-n}$, 试证明在$n\to \infty$时$f_n$几乎处处收敛到$f$.
    \item 设$f\in L(\realset)$且$\int_{\realset} f \neq 0$, $a$是一个确定的实数,证明
    \begin{equation*}
        x\mapsto \frac{1}{2x}\int_{a-x}^{a+x}f\quad\notin L(\realset).
    \end{equation*}
    \item 设$f\in L(\realset)$, $a>0$,证明
    \begin{equation*}
        \lim_{n\to \infty}n^{-a}f(nx)=0,\quad\text{a.e. }x\in \realset.
    \end{equation*}
    \item 已知曲面$2x^2+ay^2+2z^2+2xy+2xz+2yz=3$经正交变换可化为椭球面$u^2+v^2+bw^2=3$, 求$a,b$的值。
    \item 证明复矩阵的Jordan-Chevalley分解:任何$A\in M_n(\complexset)$都可以分解为$B+C$的形式,其中
    \begin{enumerate}
        \item $B$可对角化,
        \item $C$幂零,
        \item $BC=CB$,
        \item $B,C$都是$A$的多项式,
    \end{enumerate}
    并且满足(a)-(c)的分解是唯一的。
\end{enumerate}
\section*{2024.11.14}
如无特别说明,所有函数均取实值。
\begin{enumerate}
    \item 设有正数列$(a_n)_n$, 且存在$\alpha>0$使得$\sum_{n} a_n^\alpha$收敛,问:$\sum_{n}a_n/n$是否一定收敛?
    \item 设$f\in C^\infty(\realset)$, 存在函数$\varphi\colon\realset\to\realset$作为函数列$\bigl\{f^{(n)}\bigr\}_n$在任何有限区间上的一致极限,求解$\varphi$.
    \item 求和$\sum_{n=1}^\infty \Bigl(\sum_{k=1}^{n}\frac{1}{k}\Bigr) x^n$.
    \item Maclaurin展开$x\mapsto \ln \bigl(x+\sqrt{1+x^2}\bigr)$.
    \item 用幂级数的乘法说明$\exp (x+y)=(\exp x)(\exp y)$.
    \item 求$S$在$\realset^2$中的导集$S'$与闭包$\overline S$, 其中$S:=\{(x,\sin (1/x))\in\realset^2\colon 0<x\leqslant1\}$.\\
    $\overline S$是赫赫有名的``拓扑学家的正弦曲线'',试说明它在$\realset^2$中是连通但不道路连通的。
    \item 设$f\in C([a,b]\times[c,d])$,函数列$(\varphi_n)_n$于$[a,b]$上一致收敛,且在$[a,b]$上逐点成立$c\leqslant \varphi_n\leqslant d$. 试证明$\bigl\{x\mapsto f(x,\varphi_n(x))\bigr\}_n$在$[a,b]$上一致收敛。
    \item 证明:$\realset^n$中的有界开集$G$上的一致连续函数一定可以延拓成$\overline G$上的一致连续函数。
    \item 证明单位球面$x^2+y^2+z^2=1$和锥面$x^2+y^2=cz^2$正交(在任何交点处的切平面相互垂直),其中$c>0$是常数。试几何地解释这个现象。
    \item 用尽可能多的方法解$\sup_{A}f$和$\inf_{A}f$, 其中$A\colon x+y-1=0$.
    \item 用尽可能多的方法解$n$元实二次型在$\realset^n$中的单位球面($l_2$范数意义下)上的最值。计算$x^2+xy+y^2\leqslant 1$的面积。
    \item 计算$\int_L \sqrt{2y^2+z^2}\differential s$, 其中$
        L$是$\realset^3$中$x=y$和$x^2+y^2+z^2=a^2(a>0)$的交线.
    \item 在平面直角右手坐标系中,原点处有一质量为$M$的质点A,并有一个质量为$m$的质点B沿$\{(x,y)\in [0,\infty)^2\colon (x/a)^2+(y/b)^2=1\}$从$(a,0)$无折返地运动到$(0,b)$,问在这一过程中A对B的万有引力所做的功。A对B的万有引力的方向为平面向量$\overrightarrow{BA}$的方向,大小为$GMmr^{-2}$, 其中$G$是正常量,$r$是A与B之间的距离。
    \item 计算$\{(a(\cos t)^3, a(\sin t)^3)\in \realset^2\}$在$\realset^2$上所围图形的面积,其中$a>0$。
    \item 求边长为$a$、密度均匀(设为$\rho$)的立方体关于其任意棱边的转动惯量。
    \item 已知$a+\sqrt{a^2-y^2}=y\mathrm{e}^u$, $au=x+\sqrt{a^2-y^2}$, $a>0$, 求$\differential y/\differential x$和$\differential^2 y/\differential x^2$.
    \item 把偏微分方程$(x+y)z_x-(x-y)z_y=0$换成以$u,v$为自变量的形式,其中$u=\ln\sqrt{x^2+y^2}$, $v=\arctan (y/x)$.
    \item 求曲面$x^2+2y^2+3z^2=21$的切平面,使它平行于平面$x+4y+6z=0$.
    \item 计算$\iint_D xy^2 \differential\sigma$, 其中$D$是由$y^2=4x$, $x-y=1$, $x+y=1$所围成的$\realset^2$中的有界区域。
    \item 计算$\iiint_V \frac{\differential x\differential y\differential z}{(1+x+y+z)^3}$, 其中$V$是由$x+y+z=1$与三个坐标面所围成的体积。
\end{enumerate}
\section*{2024.07.04 数学分析开局测试}
约定$\naturalset$表示所有非负整数的集合,$\positiveinteger$表示所有正整数的集合。设$n,m\in\positiveinteger$, 对于可微映射$f\colon \realset^n\to\realset^m$, $f'$表示$f$的Jacobi矩阵。实值函数$f$在区间$I$上是凸函数,意思是对于任何$x,y\in I,\lambda+\mu=1,\lambda,\mu\geqslant 0$, 都有$f(\lambda x+\mu y)\leqslant \lambda f(x)+\mu f(y)$.
\begin{enumerate}
    \item 叙述实数系基本定理。
    \item 证明Toeplitz定理:设有一族非负实数$\left\{\{t_{n,k}\}_{k=1}^n\right\}_{n=1}^\infty$满足$\sum_{k=1}^{n}t_{n,k}=1,\lim_{n\to\infty}t_{n,k}=0$, 另有实数列$\{a_n\}_{n=1}^\infty$收敛到$a\in \realset$, 那么就有\begin{equation*}
        \lim_{n\to \infty}\sum_{k=1}^{n}t_{n,k}a_k=a.
    \end{equation*}
    \item 设实数列$\{a_n\}_{n=1}^{\infty}$的每一项都大于0, 试比较\begin{align*}
        &\liminf_{n\to\infty}\sqrt[n]{a_n},&&\limsup_{n\to\infty}\sqrt[n]{a_n},&&\liminf_{n\to\infty}\frac{a_{n+1}}{a_n},&&\limsup_{n\to\infty}\frac{a_{n+1}}{a_n}
    \end{align*}的大小关系。
    \item 有限区间上的一致连续函数是否一定有界?
    \item 设$-\infty<a<b<\infty$, $f\colon [a,b]\to\realset$为凸函数。试证明,如果存在$c\in(a,b)$使得$f(a)=f(c)=f(b)$, 那么$f$恒取常值。
    \item 设$I$是一个开区间,$f$是$I$上的凸函数。试证明:
    \begin{enumerate}
        \item 在$I$上存在递增的左导函数$f'_-$和右导函数$f'_+$, 并且在$I$上逐点地有$f'_-\leqslant f'_+$.
        \item 设$x\in I$. 若$f'_+$在$x$处左连续,或$f'_-$在$x$处右连续,则$f$在$x$处可导。
        \item $f$在$I$上局部Lipschitz, 即在任何$I$的紧子集上Lipschitz.
    \end{enumerate}
    \item Dirichlet函数和Riemann函数在$[0,1]$上的Riemann可积性如何,Lebesgue可积性又如何?可积的情况中,积分值分别是多少?
    \item 导出带积分余项的Taylor公式:设$n\in\naturalset$, $f\in C^{n+1}(a,b)$, $x_0\in (a,b)$, 则有\begin{equation*}
        f(x)=\sum_{k=0}^{n}\frac{f^{(k)}\left(x_0\right)}{k!}\left(x-x_0\right)^k+\frac{1}{n!}\int_{x_0}^x (x-t)^n f^{n+1}(t)\differential t.
    \end{equation*}
    \item 对于$n\in \naturalset$计算积分\begin{equation*}
        \int_0^{\pi/2}(\sin x)^n\differential x.
    \end{equation*}
    \item 计算极限\begin{equation*}
        \lim_{n\to \infty}\frac{(n!)^2 2^{2n}}{(2n)!\sqrt{n}}.
    \end{equation*}
    \item 写出Young不等式、H\"older不等式和Minkowski不等式。
    \item 设$n\in\positiveinteger$. 给出$\realset^n$中开集、闭集、导集、闭包、完全集、紧集、列紧集、连通集、道路连通集、凸集、区域的定义。
    \item 设$n\in\positiveinteger$. 在$\realset^n$的所有子集中,是否存在既不是开集又不是闭集的集合?是否存在既开又闭的集合?有哪些既开又闭的集合?这体现了$\realset^n$的什么性质?
    \item 设$n\in\positiveinteger$. $\realset^n$中的区域是否一定是道路连通的?
    \item 设$n,m\in\positiveinteger$. 问:$\realset^n\to\realset^m$的连续映射\\
    \begin{enumerate*}
        \item 是否一定把开集映成开集?\hspace*{5em}
        \item 是否一定把闭集映成闭集?\\
        \item 是否一定把紧集映成紧集?\hspace*{5em}
        \item 是否一定把连通集映成连通集?
    \end{enumerate*}
    \item 设$n,m\in\positiveinteger$. 试用开集来刻画$\realset^n\to\realset^m$的连续映射。又问:$\realset^n\to\realset$的Lebesgue可测映射有没有类似的刻画?
    \item 设\begin{equation*}
        \left\{\begin{aligned}
            &u^2-v\cos xy+w^2=0,\\
            &u^2+v^2-\sin xy+2w^2=2,\\
            &uv-(\sin x)(\cos y)+w=0.
        \end{aligned}\right.
    \end{equation*}
    在$(x,y)=(\pi/2,0),(u,v,w)=(1,1,0)$处计算Jacobi矩阵
    \begin{equation*}
        \frac{\partial(u,v,w)}{\partial(x,y)}.
    \end{equation*}
    \item (多元函数的中值定理)设$n\in\positiveinteger$, 凸区域$D\subseteq \realset^n$, 函数$f\colon D\to\realset$可微,则对任何两点$a,b\in D$, 在这两点的连线上存在一点$\xi$, 使得\begin{equation*}
        f(b)-f(a)=f'(\xi)(b-a).
    \end{equation*}
    \item (拟微分平均值定理)设$n,m\in\positiveinteger$, 凸区域$D\subseteq \realset^n$, 函数$f\colon D\to\realset^m$可微,则对任何两点$a,b\in D$, 在这两点的连线上存在一点$\xi$, 使得\begin{equation*}
        \left\|f(b)-f(a)\right\|_2\leqslant\left\|f'(\xi)\right\|_F\left\|(b-a)\right\|_2,
    \end{equation*}
    其中$\|\cdot\|_2$表示列向量的2范数(也即由Euclid空间上的标准内积诱导的范数),$\|\cdot\|_F$表示矩阵的Frobenius范数。
    \item 叙述Newton-Leibniz公式、Green公式、Gauss公式和Stokes公式。对于有限区间上的Lebesgue积分,Newton-Leibniz公式何时成立?
    \item 尽可能多地说出数项级数判敛的方法。
    \item 尽可能多地说出积分与极限换序的条件。
    \item 设有连续函数列$\{f_n\}_{n=1}^\infty\colon [a,b]\to\realset$. 试证明,如果$\{f_n\}_{n=1}^{\infty}$在$[a,b)$内逐点收敛,但$\{f_n(b)\}_{n=1}^\infty$发散,那么$\{f_n\}_{n=1}^{\infty}$在$[a,b)$上不可能一致收敛。
    \item 说说Fourier系数定义式的思想。
    \item 证明Riemann-Lebesgue引理:设$f$是有限闭区间$[a,b]$上的(常义)Riemann可积实值函数,那么\begin{equation*}
        \lim_{\lambda\to +\infty}\int_a^b f(x)\cos (\lambda x)\differential x=0.
    \end{equation*}
\end{enumerate}